\documentclass[a4paper, 12pt]{article}
\usepackage{graphicx}
\usepackage{xcolor}
\usepackage{scalerel}
\usepackage{fancyhdr}
\usepackage{tikz, pgfplots, tkz-euclide,calc}
\usepackage{geometry}
 \geometry{
        total = {160mm, 237mm},
        left = 25mm,
        right = 35mm,
        top = 30mm,
        bottom = 30mm,
      }
      
\usepackage{tabularx}
\usepackage{fancyhdr}
\usepackage{graphicx}
\usepackage{amssymb}
\usepackage{amsmath}
\usepackage{multicol}
\usepackage{graphicx}
\usepackage{wrapfig}
\usepackage{enumitem}
\usepackage{lastpage}
\usepackage{listings}
\usepackage{transparent}
\usepackage{cancel}
\usepackage{setspace}

\definecolor{pgray}{rgb}{0.5,0.5,0.5}
\definecolor{pblue}{rgb}{0.13,0.13,1}
\definecolor{pgreen}{rgb}{0,0.5,0}
\definecolor{pred}{rgb}{0.9,0,0}
\definecolor{pgrey}{rgb}{0.46,0.45,0.48}
\definecolor{pcyan}{HTML}{D4EFFC}
\definecolor{lblue}{HTML}{00AEEF}
\definecolor{input}{HTML}{AAE1FA}
\definecolor{bg}{rgb}{0.95, 0.95, 0.92}
\definecolor{vscode}{HTML}{282A36}

\onehalfspacing
\newcommand{\blue}[1]{\textcolor{blue}{#1}}
\lstdefinestyle{output}{
    language=Java,
    backgroundcolor=\color{vscode},
    basicstyle=\small\ttfamily\color{white},
    frame=single,
    escapeinside={(*}{*)},
    showspaces=false,
    showtabs=false,
    breaklines=true,
    showstringspaces=false,
    breakatwhitespace=true,
    }

\lstdefinestyle{standard}{language=Java,
    showspaces=false,
    showtabs=false,
    breaklines=true,
    showstringspaces=false,
    breakatwhitespace=true,
    commentstyle=\color{pgray},
    keywordstyle=\color{pblue},
    stringstyle=\color{pgreen},
    basicstyle=\small\ttfamily,
    frame=single,
    backgroundcolor=\color{bg},
    escapeinside={(*}{*)},}


\title{Tugas Pertemuan 1}
\author{Dhanar A \& Fajar A}
\date{11 September 2024}

\begin{document}

\maketitle

\section*{Problems}
\begin{enumerate}
    \item Buatlah sebuah program yang outputnya menampilkan biodata dengan ketentuan 
    \begin{itemize}
        \item Biodata mencakup nama lengkap \& panggilan , nrp , angkatan , asal , TTL , Dosen Wali
        \item Deklarasikan setiap variabel dengan tipe data yang sesuai 
        \item Gunakan \blue{System.out.println()} untuk menampilkan variabel dan nilai yang telah di-deklarasikan sebelumnya
    \end{itemize}
    Contoh Output : 
    \begin{lstlisting}[style=output]
                ----- Biodata Mahasiswa -----
    Nama Lengkap    : Fajar Anwar
    Nama Panggilan  : Anwar
    NRP             : 5002221075
    Angkatan        : 2022
    Asal            : Surabaya
    TTL             : 12 September 2024
    Dosen Wali      : Prof Sigma
    \end{lstlisting}

    \newpage
    \item Diberikan program seperti berikut 
    \begin{lstlisting}[style=standard]
public class Week1_no2 {
    public static void main (String []args) {
    String nama = "Nar Dios";
    int nrp = 123456789;
    String asal = "Surabaya";
    int angkatan = 2024;
    String TTL = "12 Januari 2024";
    String dosenWali = "Prof Sigma";
    
    System.out.println("-----Biodata Mahasiswa-----");
    System.out.println(Nama :  + nama);
    System.out.println("NRP : " + nrp);
    System.out.println("Asal: " + asal);
    System.out.println("Angkatan : " + angkatan);
    System.out.println("TTL: " + TTL);
    System.out.println("Dosen Wali: " + dosenWali);
    System.out.println("Alamat: " + alamat); 
    }
}
    \end{lstlisting}
    Perbaiki kesalahan dari program berikut dengan \underline{menambahkan} apapun agar program dapat di running , kemudian tampilkan outputnya 
\end{enumerate}

\end{document}
